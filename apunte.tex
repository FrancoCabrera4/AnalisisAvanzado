\documentclass[12pt]{article} 
\usepackage{amsfonts}
\usepackage{amsthm}
\usepackage{amsmath}
\usepackage{amssymb}
\newtheorem{theorem}{Teorema}[section]
\newtheorem{definition}{Definición}[section]
\newtheorem{prop}{Propiedad}[section]
\newtheorem{notacion}{Notación}[section]
\newtheorem{colorario}{Colorario}[section]
\begin{document}


\section{Introducción}


\section{Preliminares}

\begin{definition}
    Sea $A \in \mathbb{R}$ un conjunto no vacío. Se dice que $c \in \mathbb{R}$ es una cota superior de $A$ si $c \ge x, \forall x \in A$. Un conjunto acotado superiormente es aquel que tiene una cota superior.
\end{definition}

\begin{definition}
    Sea $A \in \mathbb{R}$ un conjunto no vacío y acotado superiormente. Se dice que $s \in \mathbb{R}$ es supremo de $A$ si cumple:
    \begin{itemize}
        \item $s$ es cota superior de $A$
        \item $\forall t$ cota superior de $A$, $t \ge s$
    \end{itemize}
\end{definition}

\begin{prop}
    Sea $A \in \mathbb{R}$ un conjunto no vacío y acotado superiormente. $s \in \mathbb{R}$ es supremo de $A$ si y solo si cumple:
    \begin{itemize}
        \item $s$ es cota superior de A
        \item $\forall \epsilon > 0, \exists a \in A : a > s - \epsilon$
    \end{itemize}
\end{prop}
\begin{proof}
    Si $s$ es supremo entonces ya se verifica la primera de las dos propiedades. Para la segunda procedo por el absurdo, supongo que $\exists \epsilon > 0, \forall a \in A : a \le s - \epsilon$, por lo tanto se cumple que $t = s - \epsilon$ es cota superior de $A$, pero a su vez $t < s$, donde $s$ era supremo de $A$, absurdo, y por lo tanto no existe dicho $\epsilon$ y se demostró lo pedido.

    Para la otra dirección de la equivalencia también tenemos la primera condición dada. Supongo ahora que $\exists t < s$ tal que $t$ es cota superior. Por hipótesis si se fija $\epsilon = s - t, \exists a \in A : a > s - \epsilon = s - (s - t) = t \Rightarrow a > t$, pero se tiene que $t$ es cota superior, absurdo. Y por ende se concluye que $s$ es supremo de $A$.
\end{proof}
\begin{prop}
    Sea $A \in \mathbb{R}$ un conjunto no vacío y acotado superiormente. Luego $s \in \mathbb{R}$ es supremo de $A$ si y solo si:
    \begin{itemize}
        \item $s$ es cota superior de $A$
        \item $\exists (a_n) \subset A : a_n \rightarrow s$
    \end{itemize}
\end{prop}
\begin{proof}
    Si $s$ es supremo ya se cumple la primera propiedad y usando la equivalencia se tiene que $\forall n \in \mathbb{N}, \exists a_n : a_n > s - \frac{1}{n}$, de donde es claro que la sucesión dada por estos términos cumple $a_n \rightarrow s$. 

    Para la otra dirección, de forma análoga ya se tiene la primera propiedad. Supongo que el supremo no es $s$, sea $t$ el supremo, luego se tiene que $t < s$ y se puede definir $\epsilon = s - t$ y por definición de límite se cumple que $\exists n_0 \in \mathbb{N} : n \ge n_0 \Rightarrow |s - a_n | < \epsilon$ y por lo tanto se cumple que $a_n > s - \epsilon = s - (s - t) = t$, pero $t$ es el supremo de $A$, absurdo. Y se demostró lo pedido.
\end{proof}

\section{Cardinalidad}

En esta sección vamos a dedicar un tiempo a pensar y explorar la noción de tamaño o cardinalidad de conjuntos en los que habitualmente no pensamos de esta forma, estos son los conjuntos no finitos. Cuando un conjunto es finito la cantidad de elementos que tiene es naturalmente su cardinal y de esta forma podemos notar que si $A = \{1, 2, 3\}$ entonces $\#A = 3$.

Sin embargo este approach no funciona con los conjuntos no finitos, porque resultaría que todos tendrían cardinal infinito y toda nuestra exploración termina ahí. Pero al encontrar un bloqueo no hay que rendirnos y en este caso una idea muy útil y algo intuitiva es que si no podemos contar algo, podemos intentar contar algo que sea parecido o incluso mejor tenga la misma cantidad de elementos y sí sea facil de contar. Y esa es la idea con la que vamos a empezar, Georg Cantor entendió que no podía contar directamente los elementos de un conjunto, pero al menos sí podía comparlos entre sí.

\begin{definition}
    Sean $X, Y$ dos conjuntos. Decimos que son coordinables (o equipotentes, o de igual cardinal) si existe $f : X \rightarrow Y$ tal que $f$ es biyectiva. Y se suele notar $X \sim Y$
\end{definition}
\begin{prop}
    La relación $\sim$ es una relación de equivalencia.
\end{prop}
\begin{proof}
    Sean $X, Y, Z$ conjuntos tales que $X \sim Y$ y $Y \sim Z$. La identidad es biyectiva para todo conjunto por lo tanto la relación es reflexiva. Si $f : X \rightarrow Y$ es biyectiva, entonces tiene inversa y $f^{-1} : Y \rightarrow X$ es biyectiva y se sigue que la relación es simétrica. Si $g : Y \rightarrow Z$ es biyectiva se tiene que la composición $z = g \circ f, z : X \rightarrow Z$ es biyectiva y por tanto la relación es transitiva. Teniendo estas tres propiedades se sigue que la relación es de equivalencia.
\end{proof}
\begin{definition}
    Un conjunto $A$ se dice numerable si $A \sim \mathbb{N}$. Un conjunto $A$ se dice contable si $A$ es finito o numerable.
\end{definition}
\begin{definition}
    Decimos que $\#X \le \#Y$ si existe $f : X \rightarrow Y$ inyectiva. Luego, decimos que $\#X < \#Y$ si $\#X \le \#Y$, pero $\#X \neq \#Y$ 
\end{definition}

\begin{theorem}
    
\end{theorem}
    Sea $X$ un conjunto, entonces $\#X < \#P(X)$ 
\begin{proof}
    Procedemos por el absurdo y supongamos que esto no es cierto. Y como es claro que la función identidad es inyectiva de $X \rightarrow P(X)$, entonces $X \sim P(X)$. Por lo tanto existe una function biyectiva entre los conjuntos, sea esta $f: X \rightarrow P(X)$. Considero el conjunto $A := \{ a \in X : a \notin f(a) \}$, como claramente $A \in P(X)$ tiene sentido evaluar $f^{-1}(A) = B$. 

    Ahora existen 2 casos, o bien $B \in A$ o $B \notin A$. 

    Caso 1° $B \in A$, entonces $B \notin f(B) = A \Rightarrow B \notin A$. Absurdo

    Caso 2° $B \notin A$, entonces $B \in f(B) = A \Rightarrow B \in A$. Absurdo

    Pero $f$ era una funcion biyectiva, pero sin embargo $f(B)$ no está definida. Por lo tanto la supocisión inicial era incorrecta y se concluye que $\#X < \#P(X)$
\end{proof}

La idea de este conjunto y demostración está acuñada bajo el término de la paradoja de Russel

\begin{theorem}
    Teorema de Schroder-Berstein. Si existen $f: X \rightarrow Y, g: Y \rightarrow X$ inyectivas, entonces existe $h: X \rightarrow Y$ biyectiva.
\end{theorem}

Este teorema va a ser fundamental porque en muchísimos casos encontrar una función biyectiva entre dos conjuntos va a ser bastante complicado, mientras que encontrar un par de funciones inyectivas como las del enunciado no va a ser tan difícil. La prueba es un poco difícil de adquirir a la primera, pero leyendola un par de veces sale.

\begin{proof}
    Sea $A_1 = X \setminus g(Y)$ e inductivamente se define $A_n = g(f(A_{n-1}))$. Afirmo que estos conjuntos son disjuntos dos a dos, para eso procedo por el absurdo y asumo que no es así, entonces $\exists j, i : x \in A_j, x \in A_i$, sin perdida de generalidad asumo que $i < j$. Como $g, f$ son ambas inyectivas se sigue que $\exists y : y \in f(A_{j-1}), y \in f(A_{i-1})$, donde a su vez $\exists k : k \in A_{j-1}, k \in A_{i-1}$. Por lo tanto llevando este argumento a su caso base se tiene que $\exists m : m \in A_1, m \in A_{j + 1 - i}$, pero si $m \in A_1 \Rightarrow m \notin g(Y)$, donde a su vez si $m \in A_{j + 1 - i} \Rightarrow m \in g(Y)$, absurdo. 

    Por lo tanto sea $A = \cup A_n, B = \cup f(A_n)$, se sigue que $f: A \rightarrow B$ es una función biyectiva. Luego sea $A' = X \setminus A, B' = Y \setminus B$, es claro que $g: Y \rightarrow g(Y)$ es biyectiva por definición, entonces $g(B') = g(Y \setminus B) = g(Y) \setminus g(B) =  A'$

    Finalmente se tiene que la función $h: X \rightarrow Y$ es biyectiva y cumple lo pedido 
    \[
       h(x) := \begin{cases}
           f(x) & x \in A \\
           g^{-1}(x) & x \in X \setminus A
       \end{cases} 
    \]

\end{proof}

Revisar esta última demo

\begin{prop}
    Sea $\{ A_n \}$ una familia numerable de conjuntos numerables, entonces $A = \cup A_n$ es un conjunto numerable.
\end{prop}

\begin{theorem}
    Teorema de diagonalización de Cantor
\end{theorem}

\begin{prop}
    Hipótesis del continuo de Cantor
\end{prop}

\section{Espacios Métricos}

\begin{definition}
   Espacio métrico. Sea $E$ un conjunto. Una función $d: E \times E \rightarrow \mathbb{R}$ se llama una métrica o una distancia sobre $E$ si se cumple: 
   \begin{itemize}

       \item $d(x, y) = 0 \iff x = y$
       \item $d(x, y) = d(y, x), \forall x, y \in E$
       \item $d(x, y) \leq d(x, z) + d(z, y), \forall x, y, z \in E$
       
   \end{itemize}
\end{definition}

\begin{definition}
    Se define a la distancia euclídea $d_2$ para $x, y \in \mathbb{R}^n$ como: 
    \[
        d_2(x, y) = (\sum_{i=1}^{n} (x_i - y_i)^{2})^{\frac{1}{2}} = \parallel x - y \parallel_2
    \]
\end{definition}

\begin{definition}
    De forma más general se va a definir la norma n en $\mathbb{R}^n$ como:
    \[
        d_n(x, y) = (\sum_{i=1}^{n} (x_i - y_i)^{n})^{\frac{1}{n}} = \parallel x - y \parallel_n
    \]
    Con la particularidad de que también se define la norma infinito como:
    \[
        d_{\infty}(x, y) = max_{1\leq i \leq n} \{|x_i - y_i|\}
    \]
    Siendo esto así porque este es el límite de las distancias cuando n se va a infinito
\end{definition}

\begin{definition}
    Dado un intervalo cerrado $[a, b] \subset \mathbb{R}$, se denota $C([0, 1])$ al conjunto de todas las funciones $f: [a, b] \rightarrow \mathbb{R}$ al conjunto de todas las funciones $f: [a, b] \rightarrow \mathbb{R}$ continuas.
\end{definition}

\begin{definition}
    Este conjunto tiene dos distancias intuitivas: la infinito y la uno. Se definen como:
    \[
        d_{\infty}(f, g) = sup_{a\leq t\leq b} |f(t) - g(t)|
    \]
    \[
        d_1(f, g) = \int_{a}^{b} |f(t) - g(t)|
    \]
\end{definition}

\begin{definition}
    Sea $E$ un conjunto cualquiera. Definimos la distancia discreat en $E$ como:
    \[
        \delta(x, y) = \begin{cases}
            0 & x = y \\
            1 & x \neq y
        \end{cases}
    \]
\end{definition}

\begin{definition}
    Dados $x \in E$ y $r > 0$, la bola abierta de centro $x$ y radio $r > 0$ es el conjunto definido por: 
    \[
        B(x, r) = \{y \in E : d(x, y) < r\}
    \]
\end{definition}

\begin{definition}
    Dados $x\in E$ y $r > 0$, la bola cerrada de centro $x$ y radio $r$ es el conjunto:
    \[
        \bar{B}(x, r) = \{y\in E : d(x, y) \leq r \}
    \]
\end{definition}

\begin{definition}
    Sea $A \subset E$. Decimos que $x$ es un punto interior de $A$ si existe algún $r>0$ tal que $B(x, r) \subset A$
\end{definition}

\begin{definition}
    Sea $A \subset E$. El interior de $A$ es el conjunto de todos los puntos interiores de $A$, y se lo nota $A^{\circ}$
\end{definition}

\begin{definition}
    Un conjunto $G \subset E$ se dice abierto si cada punto de $G$ es un punto interior de $G$, es decir si $G = G^{\circ}$
\end{definition}

\begin{theorem}
    Sea $\{ A_n \}$ una familia de conjuntos abiertos, entonces $A = \cup A_n$ es un conjunto abierto. Además si la familia es finita, $B = \cap A_n$ también es abierto
\end{theorem}

\begin{proof}
    
\end{proof}

\begin{definition}
    Un conjunto $V \subset E$ se llama entorno de $x$ si existe un conjunto abierto $G$ tal que $x \in G \subset V$
\end{definition}

\begin{definition}
    Se dice que $x$ es un punto de adherencia del conjunto $A \subset E$ si para todo $r > 0$, $A \cap B(x, r) \neq \emptyset$
\end{definition}

\begin{definition}
    La clausura de $A \subset E$ es el conjunto $\bar{A}$ formado por todos los puntos de adherencia del conjunto $A$
\end{definition}

\begin{definition}
    Un conjunto $F$ se llama cerrado si $F = \bar{F}$
\end{definition}

\begin{theorem}
    Un conjunto $A$ es cerrado si y solo si $A^{c}$ es abierto.
\end{theorem}
\begin{proof}
    
\end{proof}
\begin{theorem}
    Sea $\{ A_n \}$ una familia de conjuntos cerrados. Si la familia es finita, entonces $A = \cup A_n$ es un conjunto cerrado.
\end{theorem}
\begin{proof}
    
\end{proof}

\begin{definition}
    Dada un conjunto $A$, el conjunto de puntos de acumulación de $A \subset E$ se denomina conjunto derivado de $A$, y se lo nota:
    \[
        A' = \{ x\in E : x \text{es punto de acumulación de} A \}
    \]
\end{definition}

\begin{theorem}
    Sea $A \subset E$, entonces $\bar{A} = A \cup A'$
\end{theorem}

\begin{definition}
    Dado $A \subset E$, se dice que $x$ es un punto de la frontera de $A$ si para todo $r>0$, se cumple:
    \[
        B(x, r) \cap A \neq \emptyset, B(x, r) \cap A^{c} \neq \emptyset
    \]
    Al conjunto de los puntos frontera se lo nota $\partial A$
\end{definition}

\begin{definition}
    Decimos que una sucesión $(x_n)_{n\in \mathbb{N}} \subset E$ converge a $x \in E$ si dado cualquier $\epsilon > 0$ existe un $n_0 \in \mathbb{N}$ tal que $d(x, x_n) < \epsilon, \forall n \geq n_0$ 
\end{definition}

\begin{definition}
    Dado un conjunto $A \subset E$, un punto $x \in A$ se dice aislado si existe $r>0$ tal que $B(x, r) \cap A = \{ x \}$
\end{definition}

\begin{definition}
Dado un conjunto $A \subset E$, se dice que es acotado si existen $x\in E, r>0$ tal que $A \subset B(x, r)$
\end{definition}

\begin{definition}
    Una sucesión $(x_n)_{n\in \mathbb{N}}$ se dice acotada si se encuentra contenida en un conjunto acotado. Es decir, $\forall n \in \mathbb{N} : a_n \in A$, donde $A$ es un conjunto acotado.
\end{definition}

\begin{definition}
    Una sucesión $(x_n)_n$ se dice de cauchy si para todo $\epsilon > 0$ existe $n_0 \in \mathbb{N}$ tal que si $n, m \geq n_0$, entonces $d(x_n, x_m) < \epsilon$
\end{definition}

\begin{theorem}
    Sea $(E, d)$ un espacio métrico y $(x_n)_n \subset E$, luego:
    \begin{itemize}
        \item Si $(x_n)_n$ es de cauchy, entonces es acotada
        \item Si $(x_n)_n$ es convergente, entonces es de cauchy
        \item Si $(x_n)_n$ es de cauchy y tiene alguna subsucesión convergente, entonces $(x_n)_n$ es convergente. 
    \end{itemize}
\end{theorem}
\begin{proof}
    
\end{proof}
\begin{definition}
    Un espacio métrico $(E, d)$ se dice completo si toda sucesión de cauchy es convergente a algún punto $x \in E$
\end{definition}

\section{Funciones Continuas}
    El primer acercamiento que diría todos tienen al concepto de funciones continuas está capturado en la frase "las funciones que no dan salto" o "se pueden dibujar sin levantar el lápiz", esto está bien para el contexto donde se presenta, pero a esta altura ya vimos lugares donde esa definición no sirve, por ejemplo una función $f: \mathbb{R} \rightarrow \mathbb{R}^2$. Por eso se trabaja con la definición de epsilon-delta, y en esta sección se van a presentar todo el resto de herramientas iniciales para trabajar con este tipo de funciones en todos los espacios métricos. 

\begin{definition}
    Una función $f: E \rightarrow E'$ se dice continua en el punto $x \in E$ si para cada $\epsilon > 0$ existe $\delta > 0 $ tal que
    \[
        y \in E, d(x, y) < \delta \Rightarrow d'(f(x), f(y)) < \epsilon
    \]
    O equivalentemente utilizando la definición topológica, para cada $\epsilon > 0$ existe $\delta > 0$ tal que 
    \[
        f(B(x, \delta)) \subset B(f(x), \epsilon)
    \]
\end{definition}

Honestamente para mi esta es la mejor equivalencia de continuidad para una función, que las sucesiones convergentes, convergen a donde uno esperaría en la imagen. 
\begin{theorem}
    Una función $f: E \rightarrow E'$ es continua en $x$ si y solo si transforma cualquier sucesión convergente a $x$ en una sucesión convergente a $f(x)$
\end{theorem} 
\begin{proof}
    $\Rightarrow)$ $f$ es continua y sea $(a_n)_{n \in \mathbb{N}} \subset E$ una sucesión tal que $\lim_{n\to\infty} a_n = a \in E$. Dado $\epsilon > 0$, como $f$ es continua en particular lo es en $x$ y por lo tanto existe $\delta > 0$ tal que $f(B(x, \delta)) \subset B(f(x), \epsilon)$. A su vez como la sucesión converge a $x$ existe un $n_0$ tal que $x_n \in B(x, \delta), \forall n \geq n_0$. Uniendo ambas partes resulta que si $n \geq n_0$ entonces $x_n \in B(x, \delta) \Rightarrow f(x_n) \in f(B(x, \delta)) \Rightarrow f(x_n) \in B(f(x), \epsilon)$ y por lo tanto $\lim_{n\to\infty}f(x_n) = f(x) \\$

    $\Leftarrow)$ Supongo que $f$ no es continua, luego $\exists \epsilon > 0 : \forall \delta > 0, f(B(x, \delta)) \not\subset B(f(x), \epsilon)$, por lo tanto para cada $n \in \mathbb{N}$ se cumple que $\exists y_n : y_n \in B(x, \frac{1}{n}), y_n \not\in B(f(x), \epsilon)$ , pero esto significa que $y_n \rightarrow x, f(y_n) \not\rightarrow f(x)$, lo cual es un absurdo por hipótesis. Por lo tanto se concluye que $f$ es continua.
\end{proof}

Incluir 2 imagenes, una con una función discontinua en el plano, y una de una superficie que muestre que toda sucesión converge a donde uno espera.

\begin{notacion}
    Dada una función $f: E \rightarrow E'$ y un conjunto $A \subset E'$ se define la preimagen de A como
    \[
        f^{-1}(A) := \{ x \in E : f(x) \in A \}
    \]
    Es importante notar que esto NO es la inversa de la función porque esta podría no existir.

\end{notacion}

Esta es la definición de adultos de continuidad y la que les gusta a los matemáticos, todavía no encontré el amor que se le tiene, pero sí hay que reconocer que es una herramienta útil.

\begin{theorem}
    Una función $f: E \rightarrow E'$ es continua si y sólo si la preimagen de todo abierto de $E'$ es un abierto en $E$.
\end{theorem}
\begin{proof}
$\Rightarrow)$ $f$ es continua y sea $A' \subset E'$ un conjunto abierto, y se define $A = f^{-1}(A')$. Sea $x \in A$, como $A'$ es un conjunto abierto y $f(x) \in A'$ entonces $\exists\epsilon$ tal que $B(f(x), \epsilon) \subset A'$. Luego como $f$ es continua $\exists\delta>0 : f(B(x, \delta)) \subset B(f(x), \epsilon) \subset A' \Rightarrow B(x, \delta) \subset A$, y por lo tanto se tiene que $x$ es un punto interior de $A$, y como $x$ era un punto arbitrario se concluye que $A$ es un conjunto abierto. \\

$\Leftarrow)$ Sea $x \in E$ un punto, es claro que $f(x) \in E'$ y dado $\epsilon>0$ considero el abierto de la bola dado por $B(f(x), \epsilon)$, luego por hipótesis, $f^{-1}(B(f(x), \epsilon)$ es un abierto y por ende dado un punto interior $x$, $\exists\delta>0 : B(x, \delta) \in f^{-1}(B(f(x), \epsilon) \Rightarrow f(B(x, \delta) \subset B(f(x), \epsilon)$ y por lo tanto $f$ es continua.
\end{proof}

\begin{theorem}
    Una función $f: E \rightarrow E'$ es continua si y solo si para todo $A \subset E$ se cumple $f(\overline{A}) \subset \overline{f(A)}$. 
\end{theorem}
\begin{proof}
    $\Rightarrow)$ Sea $y \in f(\bar{A})$, entonces $\exists x\in \bar{A} : f(x) = y$, y además por pertenecer a la clausura se tiene que $\exists (x_n)_{n\in \mathbb{N}} \subset A: x_n \rightarrow x$. Luego ya que $f$ es continua, $f(x_n) \subset f(A) : f(x_n) \rightarrow f(x)$, y nuevamente, como existe una sucesión en el conjunto que converge a un punto, se deduce que ese punto pertenece a la clausura y por ende $f(x) \in \overline{f(A)} \Rightarrow y \in \overline{f(A)} \Rightarrow f(\bar{A}) \subset \overline{f(A)}$ \\

    $\Leftarrow)$ Sea $F \subset E'$ un conjuto cerrado y sea $A = f^{-1}(F)$. Luego $\bar{A} = \overline{f^{-1}(F)} \Rightarrow f(\bar{A}) \subset \overline{f(A)} \subset \overline{F} = F$. Con esto se tiene que $f(\bar{A}) \subset F \Rightarrow f^{-1}(f(\bar{A})) \subset f^{-1}(F) \Rightarrow \bar{A} \subset A$. Por lo tanto $A$ es un conjunto cerrado, es decir la preimagen de un cerrado cualquiera es un conjunto cerrado, esto quiere decir que $f$ es continua.
\end{proof}

Con este último teorema se completan las equivalencias básicas de continuidad que van a ser las herramientas que vamos a tener a la hora de atacar problemas. Pero hay un concepto importante más que ver antes de terminar la sección y este es: continuidad uniforme. Además nos va a servir en las secciónes siguientes cuando profundicemos en sucesiones. La idea no es muy complicada, ya vimos cuando una función es continua, pero ahora vamos a ver que hay puntos donde es más "inmediatamente continua" que en otros. Por ejemplo, si pensamos en $f: (0, +\infty) \rightarrow \mathbb{R}, f(x) = \frac{1}{x}$, es claro que $f$ es continua, pero dado un epsilon, mientras más cerca del $0$ fijemos al $x$ más difícil va a ser encontrar al $\delta$ que sirva. (REVISAR TEXTO)

\begin{definition}
    Una funcion $f: E \rightarrow E'$ se dice uniformemente continua si dado $\epsilon > 0$ existe $\delta > 0$ tal que:
    \[
        d(x, y) < \delta \Rightarrow d'(f(x), f(y)) < \epsilon, \forall x \in E
    \]
    O equivalentemente, en su versión topológica:
    \[
        f(B(x, \delta)) \subset B(f(x), \epsilon), \forall x\in E
    \]
\end{definition}

\begin{prop}
   Sea $f: E \rightarrow E'$. Entonces, $f$ NO es uniformemente continua si y solo si existen $\epsilon_0$ y sucesiones $(x_n)_n, (y_n)_n \subset E$:
   \[
       d(x_n, y_n) \rightarrow 0, d(f(x_n), f(y_n)) \geq \epsilon_0
   \]
\end{prop} 

Para el manejo de esta definición la siguiente proposición es muy importante saber negar cuantificadores, porque el problema puede parecer más complicado de lo que es si no se hace la negación correcta. 

\begin{proof}
    $\Rightarrow)$ Si $f$ no es uniformemente continua, $\exists\epsilon_0 > 0, \forall\delta>0, \exists x\in E : f(B(x, \delta)) \not\subset B(f(x),\epsilon) \Rightarrow \exists y \in E : d(x, y) <\delta, d'(f(x), f(y)) \geq \epsilon_0$. Por lo tanto para cada $n \in \mathbb{N}$ se puede fijar $\delta = \frac{1}{n}$ y se obtienen dos sucesiones $(x_n), (y_n)$ que cumplen $d(x_n, y_n) < \frac{1}{n}, d(f(x_n), f(y_n)) \geq \epsilon_0$. Es decir:
    \[
        d(x_n, y_n) \rightarrow 0, d(f(x_n), f(y_n)) \geq \epsilon_0
    \]
    \\
    $\Leftarrow)$ Por hipótesis $\exists \epsilon_0 > 0, \forall\delta > 0, \exists n_0 \in \mathbb{N} : f(B(x_{n_0}, \delta)) \not\subset B(f(x_{n_0}), \epsilon)$, puesto que $\exists y_{n_0} \in B(x_{n_0}, \delta) : y_{n_0} \not\in B(f(x_{n_0}), \epsilon)$ y por lo tanto se concluye que $f$ no es uniformemente continua.
\end{proof}

\begin{definition}
    Una función $f: E \rightarrow E'$ se llama homeomorfismo si es biyectiva, continua y su inversa es continua. Consecuentemente dos espacios métricos $(E, d), (E', d')$ se dicen homeomorfos si existe un homeomorfismo $f: E \rightarrow E'$
\end{definition}

Si uno refresca rápidamente la etimología de homeomorfismo se da cuenta que se compone de dos raíces: homoios y morphe. Homoios significa igual o semejante y morphe significa forma o figura. Entonces la pregunta natural para hacerse es: ¿en qué sentido son iguales estos dos espacios? Y la respuesta es en el topológico, porque existe una correspondencia entre los conjuntos de los dos espacios.

\begin{prop}
    Si $E, E'$ son espacios métricos homeomorfos, entonces hay una correspondencia entre los abiertos de $E$ y $E'$
\end{prop}
\begin{proof}
    Si $U \subset E'$ es un conjunto abierto, entonces $V = f^{-1}(U)$ es un abierto ya que la función es continua, y además por ser biyectiva se cumple que $f(V) = U$. Análogamente, si $W \subset E$ es un abierto, entonces $Z = f(W)$ es un abierto por ser la inversa de f una función continua, y además por ser biyectiva se tiene que $f^{-1}(Z) = W$. Con esto probamos que hay una biyección entre los abiertos de $E$ y $E'$
\end{proof}

\begin{definition}
    Si $f: E \rightarrow E'$ satisface $d(x, y) = d'(f(x), f(y))$ se dice que $f$ es una isometría.
\end{definition}

Nuevamente tiene sentido preguntarnos por la etimología, pero en este caso es un poco más obvia. Iso proviene de isos, que significa igual o lo mismo y metría viene de metron, que significa medida. Por lo tanto isometría significa de igual medida. Es decir, una isometría es una función que respeta y no altera la distancia entre puntos y las de sus imagenes.

\section{Conjuntos Compactos}

Ya entendemos lo que es un espacio métrico y tenemos diversas herramientas para clasificar y entender los distintos tipos de conjuntos y como interactuan con las funciones, es decir, que propiedades tienen. En las siguientes dos secciones vamos a desarrollar en mayor profundidad dos tipos de conjuntos que son habituales y aparecen a menudo en muchas aplicaciones.

\begin{definition}
    Sea $(E, d)$ un espacio métrico. Se dice que un subconjunto $K \subset E$ es un conjunto compacto si toda sucesión en $K$ tiene una subsucesión convergente en $K$. En otras palabras, $K$ es compacto si y solo si $\forall (x_n) \subset K, \exists (x_{n_j}) \subset K : \lim_{j\to\infty} x_{n_j} = x \in K$
\end{definition}

\begin{prop}
    Sea $K \subset E$ compacto, entonces $K$ es cerrado y acotado. 
\end{prop}
\begin{proof}
    Si $K$ no es acotado dado $x_0 \in K$, luego $\forall n\in \mathbb{N}, \exists x_n : d(x_n, x_0) \geq n$ y como $(x_n) \subset K$ es una sucesión en $K$ tiene una subsucesión convergente. Sea $(x_{n_j}) \rightarrow x$. Esto significa que $\lim_{j\to\infty} d(x_{n_j}, x) = 0$ y a su vez se cumple que $n_j \leq d(x_0, x_{n_j}) \leq d(x_0, x) + d(x, x_{n_j}) \Rightarrow \lim_{j\to\infty} n_j \leq \lim_{j\to\infty} d(x, x_{n_j}) \Rightarrow +\infty \leq 0$, absurdo! Por lo tanto $K$ es acotado. \\

    Sea $(x_n) \subset K$ una sucesión convergente, luego por ser $K$ compacto $\exists (x_{n_j}): (x_{n_j}) \rightarrow x \in K$, pero como $(x_n)$ ya era convergente toda subsucesión converge al mismo límite y por ende $(x_n) \rightarrow x \in K$ y con esto se concluye que $K$ es cerrado.
\end{proof} 

\begin{theorem}
    Teorema de Heine-Borel. Un conjunto $K \subset \mathbb{R}^{m}$ es compacto si y solo si es un conjunto cerrado y acotado
\end{theorem}
\begin{proof}
    La implicación derecha es automática por la proposición demostrada antes. Así que hay que demostrar la implicación hacia la izquierda. \\
    Sea $(x_n) \subset K$, como $K$ es acotado también lo es la sucesión. Sea $x_n = (x_n^{1}, x_n^{2}, ..., x_n^{m})$, como $x_n$ esta acotado también lo esta $x_n^{i}, \forall 1 \leq i \leq m$, y en particular lo está $x_n^{1}$, así que por el teorema de Bolzano-Weierstrass existe $x_{n_j}^{1} \rightarrow x^{1}$. Así que ahora podemos considerar a $x_{n_j} = (x_{n_j}^{1}, x_{n_j}^{2}, ..., x_{n_j}^{m})$, y repitiendo el argumento resulta que $x_{n_j}^{2}$ tiene una subsucesión $x_{n_{j_l}}^{2} \rightarrow x^{2}$ y a su vez $x_{n_{j_l}}^{1} \rightarrow x^{1}$ por que toda subsucesión de una sucesión convergente converge al mismo límite. Finalmente utilizando este argumento en las $m$ coordenadas se sigue que existe una subsucesión $x_{n_j_{..._h}} \rightarrow x$ y como $K$ es cerrado se tiene que $x \in K$ y por lo tanto se concluye que $K$ es compacto.
\end{proof}

\begin{theorem}
    Sea $(E, d)$ un espacio métrico, entonces $K \subset E$ es un compacto si y solo si todo $A \subset K$ infinito tiene un punto de acumulación en $K$.
\end{theorem}
\begin{proof}
    $\Rightarrow)$ Sea $A \subset K$ un conjunto infinito, luego existe $B \subset A$ numerable y por ende una sucesión contenida en $B$ y que por tanto está contenida en $A$, es decir, $\exists (x_n) \subset A$ una sucesión de elementos distintos. Como $K$ es compacto, $\exists (x_{n_j}) \rightarrow x \in K$, y justamente porque tiene una sucesión de elementos distintos en $A$ que convergen a él, se sigue que $x$ es un punto de acumulación de $A$. \\
    $\Leftarrow)$ Sea $(x_n) \subset K$, considero al conjunto $A = \{ x_n, n \in \mathbb{N}\}$, si $A$ es finito se sigue que la sucesión es eventualmente constante y por lo tanto tiene una subsucesión constante que converge a ese mismo elemento, es decir, $\exists n_0 \in \mathbb{N} : x_n = n_0, \forall n \geq n_1, n_1 \in \mathbb{N}$ y en este caso se sigue que $K$ es compacto. Si por el contrario $A$ es infinito $\exists (x_{n_j}) \rightarrow x \in K$, donde $x$ es el punto de acumulación, y por ende en este caso también se concluye que $K$ es compacto.
\end{proof}

\begin{definition}
    Sea $(E, d)$ un espacio métrico y $K \subset E$ un conjunto. Un cubrimiento por abiertos de $K$ es una familia $(V_i)_{i\in I}$, $V_i \subset E, \forall i \in I$ de subconjuntos abiertos de $E$ tal que 
    \[
        K \subset \cup_{i\in I} V_i
    \]
\end{definition}

\begin{definition}
    Sea $(E, d)$ un espacio métrico y $K \subset E$ un conjunto. Si $(V_i)_{i\in I}$ forma un cubrimiento de $K$ y existe $i_1, i_2, ..., i_N \in I$ tal que
    \[
        K \subset V_{i_1} \cup V_{i_2} \cup ... \cup V_{i_N}
    \]
    Se dice que $(V_{i_k})_{k=1}^{N}$ es un subcubrimiento finito de $(V_i)_{i\in I}$
\end{definition}

En la sección de funciones continuas vimos que existen muchas equivalencias para la definición de continuidad en un punto para una función y esto nos dice que a priori uno podría haber empezado con la definición que más le guste y deducir el resto de equivalencias a partir de esta. En general se empieza por la del epsilon-delta porque es la que probablemente más incialmente se ve, pero nada impide cambiar eso. De la misma forma en esta sección introdujimos la compacidad como una propiedad definida por la existencia de subsucesiones convergentes, pero sin embargo, la definición más común y que más le gusta a los matemáticos es la de subcubrimiento finito por abiertos, que se presenta a continuación. No cambia nada en las construcciones que hicimos hasta ahora, pero es importante saberlo.

\begin{theorem}
    Sea $(E, d)$ un espacio métrico, entonces $K \subset E$ es compacto si y solo si todo cubrimiento de $K$ por abiertos admite un subcubrimiento finito.
\end{theorem}

\begin{proof}
    $\Rightarrow)$ Sea $(V_i)_{i\in I}$ un cubrimiento de $K$, y supongo que no admite un subcubrimiento finito. En particular considero un subcubrimiento $(V_{i_j})$ tal que $V_{i_j} \cap K \neq \emptyset, \forall i_j \in I$ y también $\forall i_j, \exists x \in V_{i_j} : x \not\in V_{i_k}, \forall k \neq j$, es decir del cubrimiento elijo un subcubrimiento donde todos los conjuntos tengan al menos algún elemento de $K$ y que no haya conjuntos contenidos en otros. Esto sigue siendo un cubrimiento de $K$ y ahora considero $(x_n) \subset K$ tal que $x_n \in V_{i_n}, x_i \neq x_j, \forall i \neq j$. Ahora como $K$ es compacto se sigue $\exists (x_{n_j}) \rightarrow x \in K$ y digamos sin perdida de generalidad que $x \in V_{i_l}$, como $V_{i_l}$ es abierto $\exists\epsilon > 0 : B(x, \epsilon) \subset V_{i_l}$ y como la sucesión converge a ese punto se sigue que $\exists n_0 : x_{n_j} \in B(x, \epsilon), \forall j \geq n_0$, absurdo! porque solo hay un punto en la sucesión que pertenece a ese conjunto. Por lo tanto el cubrimiento sí admite un subcubrimiento finito.\\
    $\Leftarrow)$ Supongamos que no fuera el caso, luego utilizando la equivalencia de compacto para subconjuntos infinitos, se tiene que $\exists A \subset K$ tal que $A$ es infinito y $A' = \emptyset$, es decir $\forall x \in K, \exists r_x : A \cap B(x, r_x)$ es finito. Ahora por hipótesis, como $K \subset \cup_{x\in K} B(x, r_x)$ existe un subcubrimiento finito tal que $K \subset \cup_{n=1}^{N} B(x_n, r_{x_n})$. Pero entonces $A = K \cap A \subset \cup_{n=1}^{N} B(x_n, r_{x_n}) \cap A = \cup_{n=1}^{N} (B(x_n, r_{x_n}) \cap A)$, pero este último conjunto es una unión finita de conjuntos finitos, es decir $\#A < \aleph_0$, absurdo! puesto que $A$ es un conjunto infinito. Finalmente se concluye que $K$ es compacto.

\end{proof}

\begin{theorem}
    Las funciones continuas preservan la compacidad. Sean $(E, d)$, $(E', d')$ espacios métricos y sea $f: E \rightarrow E'$ una función continua. Si $K \subset E$ es un conjunto compacto, entonces $f(K)$ es compacto en $E'$
\end{theorem}

\begin{proof}
    Sea $(y_n) \subset f(K)$ una sucesión, para cada $y_n, \exists x_n \in E : f(x_n) = y_n$, donde esto genera una sucesión $(x_n) \subset K$, y como $K$ es compacto existe $(x_{n_j}) \rightarrow x \in K$ y como $f$ es continua preserva la convergencia de sucesiones y se tiene que $f(x_{n_j}) \rightarrow f(x) \in f(K)$ y por lo tanto la sucesión original $(y_n)$ posee una subsucesión $(y_{n_j}) \rightarrow y = f(x) \in f(K)$, de lo que se concluye que $f(K)$ es compacto.
\end{proof}

Este resultado es muy útil combinado con el teorema de Heine-Borel, porque nos va a decir que las funciones de la pinta $f: E \rightarrow \mathbb{R}^{n}$ con $E$ un conjunto compacto van a alcanzar un minimo y un máximo 

\begin{prop}
    Sea $K \subset E$ un conjunto compacto y $f: E \rightarrow \mathbb{R}^{m}$, entonces
        \begin{itemize}
            \item $f$ es acotada en $K$, existe una bola que contiene a todos los valores que toma la función en ese conjunto
            \item $f$ alcaza su máximo y su mínimo en $k$
        \end{itemize}
\end{prop}

\begin{theorem}
    Sean $(E, d), (E', d')$ espacios métricos y sea $f: E \rightarrow E'$, si $f$ es continua y $E$ es un conjunto compacto, entonces $f$ es uniformemente continua.
\end{theorem}
\begin{proof}
    Procedo por el absurdo, supongo que $f$ no es uniformemente continua. Por lo tanto $\exists (x_n), (y_n) \subset E : d(x_n, y_n) \rightarrow 0, d'(f(x_n), f(y_n)) \geq \epsilon_0, \epsilon_0$, para algún $\epsilon_0 > 0$. Como $E$ es un compacto, $\exists (x_{n_j})$ subsucesión convergente, de forma que se sigue cumpliendo $d(x_{n_j}, y_{n_j}) \rightarrow 0$, análogamente $\exists (y_{n_{j_k}})$ convergente, de forma que $d(x_{n_{j_k}}, y_{n_{j_k}}) \rightarrow 0$, pero como ambas sucesiones convergen, por álgebra de límites podemos reemplazar por el límite de cada sucesión, es decir, $x_{n_{j_k}} \rightarrow x, y_{n_{j_k}} \rightarrow y \Rightarrow \lim_{k\to\infty} d(x_{n_{j_k}}, y_{n_{j_k}}) = 0 = d(x, y) \Rightarrow x = y$. Ahora, como $f$ es continua respeta la convergencia de sucesiones y por ende $d'(f(x_{n_{j_k}}), f(y_{n_{j_k}})) \rightarrow d'(f(x), f(y)) = 0 \geq \epsilon_0$, absurdo! Se concluye, entonces, que $f$ es uniformemente continua.
\end{proof}
\section{Espacios Normados}

Nos encontramos ahora con otro tipo especial de espacio que es muy común de ver y este es el espacio normado, que como uno esperaría va a aparecer casi siempre cuando trabajemos con $\mathbb{R}^{n}$. La característica fundamental a recordar de estos espacios es que cuentan con una noción de escalar que la norma va a sacar a fuera. 

\begin{definition}
    Sea $E$ un espacio vectorial (sobre $\mathbb{R}^{n}$ o $\mathbb{C}^{n}$. Una función $\| . \|: E \rightarrow [0, +\infty]$ es una norma si verifica las siguientes propiedades
    \begin{itemize}
        \item $\| x + y \| \leq \| x \| + \| y \|$
        \item $\| \lambda x \| \leq | \lambda | \| x \|$
        \item $\| x \| = 0 \iff x = 0$
    \end{itemize}
\end{definition}

\begin{definition}
    Un espacio vectorial con una norma $(E, \| . \|)$ se llama un espacio normado
\end{definition}

\begin{definition}
    Un espacio normado que es completo con la distancia $d(x, y) = \| x - y \|$ se llama espacio de Banach.
\end{definition}

\begin{definition}
    Dadas dos normas $\| . \|_1, \| . \|_2$ en un espacio vectorial, se dicen equivalentes si existen $c, \bar{c} > 0$ tales que
    \[
        c\| x \|_2 \leq \| x \|_1 \leq \bar{c}\| x \|_2
    \]
\end{definition}

\begin{prop}
    La equivalencia de normas es una relación de equivalencia.
\end{prop}
\begin{proof}
    Sean $\| . \|_1, \| . \|_2, \| . \|_3$ normas en un espacio métrico tales que $\| . \|_1 \sim \| . \|_2, \| . \|_2 \sim \| . \|_3$ \\
    La relación es reflexiva puesto que $1\| x \|_1 \leq \| x \|_1 \leq 1\| x \|_1$ \\
    Es simétrica puesto que si existen $c, \bar{c} > 0 : c\| x \|_2 \leq \| x \|_1 \leq \bar{c}\| x \|_2$, entonces $\frac{1}{\bar{c}}\| x \|_1 \leq \| x \|_2 \leq \frac{1}{c}\| x \|_1$ \\
    Es transitiva puesto que si existen $c, \bar{c}, c_1, \bar{c_1}$ tal que $c\| x \|_2 \leq \| x \|_1 \leq \bar{c}\| x \|_2$, $c_1\| x \|_2 \leq \| x \|_3 \leq \bar{c_1}\| x \|_2$, entonces se tiene que $\frac{c_1}{\bar{c}}\| x \|_1 \leq \| x \|_3 \leq \frac{\bar{c_1}}{c}\| x \|_1$ \\
    Por lo tanto se concluye que la relación es de equivalencia.
\end{proof}

\begin{theorem}
    Todas las normas son equivalentes en $\mathbb{R}^{n}$
\end{theorem}
\begin{proof}
    Ya vimos que la relación de equivalencia de normas es una relación, justamente, de equivalencia, así que para demostrar este teorema basta con demostrar que toda norma es equivalente a una específica y se termina la demostración. La norma elegida para esto va a ser la $\| . \|_1$ \\
Sea $\| . \|_t$ una norma arbitraria y $x = x_1e_1 + x_2e_2 + ... + x_3e_3$, luego se tiene que $\| x \|_t = \| x_1e_1 + x_2e_2 + ... + x_ne_n \|_t \leq |x_1|\|e_1\|_t + |x_2|\|e_2\|_t + ... + |x_n|\| e_n \|_t \leq max\{ e_i, 1 \leq i \leq n\}(|x_1| + |x_2| + ... + |x_n|) = K\| x \|_1$ \\
Por otro lado, defino $g: (\mathbb{R}^{n}, \| . \|_1) \rightarrow (\mathbb{R}, | . |)$, resulta que $g$ es continua puesto que $|g(x) - g(y)| = | \| x \|_t - \| y \|_t | \leq \| x - y \|_t \leq K\| x - y \|_1$. Sea $S = \{ x \in \mathbb{R}^{n} : \| x \|_1 = 1 \}$, por el teorema de Heine-Borel $S$ es compacto. Y como $f$ es continua, $f(S)$ alcanza mínimo, sea este mínimo $m$, luego si $x \in \mathbb{R}^{n} \setminus \{ 0 \} \Rightarrow \frac{x}{\| x \|_1} \in S \Rightarrow \| \frac{x}{\| x \|_1} \|_t \geq m \Rightarrow \| x \|_t \geq m\| x \|_1$ y por lo tanto $\| x \|_t \sim \| x \|_1$, el caso donde $x = 0$ es indistinto porque todas las normas, por definición, valen $0$.
\end{proof}

\begin{definition}
    Dados dos espacios vectoriales $V, W$ y una función $f: V \rightarrow W$, se dice que $f$ es un isomorfismo si $f$ es una transformación lineal biyectiva. En este caso se dice que los espacios $V, W$ son isomorfos. Este ocurre si y solo si ambos espacios tienen la misma dimensión.
\end{definition}

\begin{prop}
    Si $E$ es un espacio normado de dimensión $n \in \mathbb{N}$, entonces existen un isomorfismo lineal de $T: E \rightarrow \mathbb{R}^{n}$ y una norma $\mathbb{R}^{n}$ tal que $T$ es una isometría.
\end{prop} 

\begin{proof}
    Como los espacios tienen la misma dimensión, $\exists T: E \rightarrow \mathbb{R}^{n}$ tal que $T$ es un isomorfismo. Sea $z \in \mathbb{R}^{n}$ se define $\| z \|_0 := \| T^{-1}(z) \|_E$. Vamos a demostrar que esto define a una norma y que esta norma cumple ser una isometría. \\
    Si $z=0 \Rightarrow \| 0 \|_0 = \| T^{-1}(0) \|_E = \| = \| 0 \|_E = 0$, puesto que toda transformación lineal manda el $0$ al $0$ y toda norma evaluada en $0$ da $0$. Ademas como $T$ es una biyección no hay otro vector que tenga como imagen al $0$, es decir $\| z \|_0 = 0 \iff z = 0$. \\
    $\| \alpha z \|_0 = \| T^{-1}(\alpha z) \|_E = \| \alpha T^{-1}(z) \|_E = |\alpha|\|T^{-1}(z)\|_E = |\alpha|\|z_0\|$, es decir la función saca escalares por fuera. \\

    $\| x + y \|_0 = \|T^{-1}(x + y)\|_E = \|T^{-1}(x) + T^{-1}(y)\|_E \leq \|T^{-1}(x)\|_E + \|T^{-1}(y)\|_E \leq \|x\|_0 + \|y\|_0$, por lo que se concluye que $T$ es una norma. \\

    Finalmente veamos que es una isometría, $d_0(T(x), T(y)) = \| T(x) - T(y) \|_0 = \| T^{-1}(T(x) - T(y))\|_E = \|T^{-1}(T(x)) - T^{-1}(T(y))\|_E = \| x - y \|_E$
\end{proof}

(REVISAR ESTA DEMO)

\begin{colorario}
    Todo espacio de normado de dimensión finita es completo, es decir es un espacio de Banach
\end{colorario}

\begin{colorario}
    En un espacio normado de dimensión finita, los conjuntos cerrados y acotados son compactos.
\end{colorario}

\begin{definition}
    Sean $E, F$ dos espacios normados sobre $\mathbb{R}$. Una aplicación $T: E \rightarrow F$ se llama operador lineal continuo si:
    \begin{itemize}
        \item Es una transformación lineal (u operador lineal)
        \item Es una función continua con las métricas que definen las normas
    \end{itemize}
\end{definition}

\begin{definition}
    Decimos que un operador lineal $T: E \rightarrow F$ es acotado si existe $c > 0$ tal que $\|T(x)\|_F \leq c\|x\|_E, \forall x\in E$. Equivalentemente, $T$ es acotado si $sup_{x \in B(0, 1)} \| T(x) \|_F < \infty$
\end{definition}

\begin{theorem}
    Sean $E, F$ espacios normados, y $T: E \rightarrow F$ un operador lineal. Son equivalentes: 
    \begin{itemize}
        \item $T$ es continua en el origen
        \item $T$ es continua en algún punto
        \item $T$ es continua
        \item $T$ es uniformemente continua
        \item $T$ está acotado
    \end{itemize}
\end{theorem}

\begin{proof}
    
    Si $B_E(x, r) = \{y\in E : \| x - y \|_E < r \}$, sea $y = x + y'$, entonces $B_E(x, r) = \{y\in E : \| x - y \|_E = \| x - (x + y') \|_E = \| y' \|_E < r \} = x + B_E(0, r)$


\end{proof}

\section{Sucesiones de Funciones}

Ya a está altura uno debería estar bastante cómodo trabajando con sucesiones, debería entender las nociones de límites, las propiedades de álgebra de límites, los teoremas más imporantes, etc. Sin embargo, hasta ahora solo trabajamos con las sucesiones más intuitivas que uno entiende, las de numeritos, y después se avanzó hacia sucesiones más abstractas en espacios métricos. Sin embargo uno puede analizar sucesiones de cualquier tipo de objecto, siempre y cuando tenga la noción de distancia entre ellos, y un objeto en particular que nos va a interesar estudiar en este contexto van a ser las funciones. En los primeros momentos es un poco confuso porque tratar a las funciones como puntos en un espacio es un concepto chocante al principio.

\begin{definition}
    La sucesión $(f_n)$ de funciones de $A \rightarrow Y$ converge puntualmente a $f: A \rightarrow Y$ si para todo $x \in A$ se cumple: 
    \[
        \lim_{n\to\infty}f_n(x) = f(x)
    \]
    O equivalentemente, $\forall x \in A, \forall \epsilon > 0, \exists n_0 \in \mathbb{N} : d'(f_n(x), f(x)) < \epsilon, \forall n \geq n_0$. Y se suele notar $f_n \rightarrow f$
\end{definition}

\begin{definition}
    La sucesión $(f_n)$ de funciones de $A \rightarrow Y$ converge uniformemente a $f: A\rightarrow Y$ si dado $\epsilon > 0$, existe $n_0 \in \mathbb{N}$ tal que si $n\geq n_0$ se tiene que:
    \[
        d'(f_n(x), f(x)) < \epsilon
    \]
    Para todo $x \in A$. O equivalentemente $\forall \epsilon > 0, \exists n_0 \in \mathbb{N}, \forall x \in A : d'(f_n(x), f(x)) < \epsilon, \forall n \geq n_0$. Y se suele notar $f_n \rightrightarrows f$
\end{definition}

(FUNDAMENTAL INCLUIR IMAGENES DE EJEMPLO ACÁ)

\section{Teoría de la medida}

\section{Integral de Lebesgue}

\end{document}

